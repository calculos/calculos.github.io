% Options for packages loaded elsewhere
\PassOptionsToPackage{unicode}{hyperref}
\PassOptionsToPackage{hyphens}{url}
\PassOptionsToPackage{dvipsnames,svgnames*,x11names*}{xcolor}
%
\documentclass[
]{krantz}
\usepackage{lmodern}
\usepackage{amssymb,amsmath}
\usepackage{ifxetex,ifluatex}
\ifnum 0\ifxetex 1\fi\ifluatex 1\fi=0 % if pdftex
  \usepackage[T1]{fontenc}
  \usepackage[utf8]{inputenc}
  \usepackage{textcomp} % provide euro and other symbols
\else % if luatex or xetex
  \usepackage{unicode-math}
  \defaultfontfeatures{Scale=MatchLowercase}
  \defaultfontfeatures[\rmfamily]{Ligatures=TeX,Scale=1}
\fi
% Use upquote if available, for straight quotes in verbatim environments
\IfFileExists{upquote.sty}{\usepackage{upquote}}{}
\IfFileExists{microtype.sty}{% use microtype if available
  \usepackage[]{microtype}
  \UseMicrotypeSet[protrusion]{basicmath} % disable protrusion for tt fonts
}{}
\makeatletter
\@ifundefined{KOMAClassName}{% if non-KOMA class
  \IfFileExists{parskip.sty}{%
    \usepackage{parskip}
  }{% else
    \setlength{\parindent}{0pt}
    \setlength{\parskip}{6pt plus 2pt minus 1pt}}
}{% if KOMA class
  \KOMAoptions{parskip=half}}
\makeatother
\usepackage{xcolor}
\IfFileExists{xurl.sty}{\usepackage{xurl}}{} % add URL line breaks if available
\IfFileExists{bookmark.sty}{\usepackage{bookmark}}{\usepackage{hyperref}}
\hypersetup{
  pdftitle={Cálculo},
  pdfauthor={ASG},
  colorlinks=true,
  linkcolor=Maroon,
  filecolor=Maroon,
  citecolor=Blue,
  urlcolor=Blue,
  pdfcreator={LaTeX via pandoc}}
\urlstyle{same} % disable monospaced font for URLs
\usepackage{longtable,booktabs}
% Correct order of tables after \paragraph or \subparagraph
\usepackage{etoolbox}
\makeatletter
\patchcmd\longtable{\par}{\if@noskipsec\mbox{}\fi\par}{}{}
\makeatother
% Allow footnotes in longtable head/foot
\IfFileExists{footnotehyper.sty}{\usepackage{footnotehyper}}{\usepackage{footnote}}
\makesavenoteenv{longtable}
\usepackage{graphicx,grffile}
\makeatletter
\def\maxwidth{\ifdim\Gin@nat@width>\linewidth\linewidth\else\Gin@nat@width\fi}
\def\maxheight{\ifdim\Gin@nat@height>\textheight\textheight\else\Gin@nat@height\fi}
\makeatother
% Scale images if necessary, so that they will not overflow the page
% margins by default, and it is still possible to overwrite the defaults
% using explicit options in \includegraphics[width, height, ...]{}
\setkeys{Gin}{width=\maxwidth,height=\maxheight,keepaspectratio}
% Set default figure placement to htbp
\makeatletter
\def\fps@figure{htbp}
\makeatother
\setlength{\emergencystretch}{3em} % prevent overfull lines
\providecommand{\tightlist}{%
  \setlength{\itemsep}{0pt}\setlength{\parskip}{0pt}}
\setcounter{secnumdepth}{5}
\usepackage{booktabs}
\usepackage{longtable}
\usepackage[bf,singlelinecheck=off]{caption}
\usepackage[scale=.8]{sourcecodepro}

%%%%%%%%%%%%%%%%%%%%%%%%%%%%%%%%%%%
%% Preámbulo usual de ASG (hace falta corregir duplicidades)

\usepackage{verbatim}
%%Los dos paquetes siguientes sirven para usar documentos de una sola ecuació\'on y para importarlos
\usepackage[subpreambles=true]{standalone}
\usepackage{import}
%%%Para m\'as detalles ver
%% https://www.overleaf.com/learn/latex/Multi-file_LaTeX_projects#The_standalone_package

%\usepackage{movie15}%para insertar videos
%ver https://tex.stackexchange.com/questions/7602/how-to-add-a-gif-file-to-my-latex-file

\usepackage{xcolor}
%\usepackage[utf8]{inputenc}
\usepackage[spanish]{babel}
\decimalpoint %para poner el punto decimal (en lugar de la coma decimal)
\usepackage{amsmath}
\allowdisplaybreaks% para permitir rompimiento de ecuaciones en p\'aginas distintas
% ver http://tex.stackexchange.com/questions/51682/is-it-possible-to-pagebreak-aligned-equations
% para m\'as detalles
\usepackage{amsthm}
%\swapnumbers%%%%%Para invertir el orden # Teorema en lugar de Toerema #, se corresponde con el paquete amsthm
\usepackage{dsfont}
\usepackage{amsfonts}
\usepackage{amssymb}%
\setcounter{MaxMatrixCols}{30}%
\usepackage{multicol}
\usepackage{graphicx}
\usepackage{cases}
\providecommand{\U}[1]{\protect\rule{.1in}{.1in}}

\setlength{\topmargin}{-1.3in}
\setlength{\textheight}{8.75in}
\setlength{\headheight}{0.75in}
\setlength{\oddsidemargin}{0.0in}
\setlength{\evensidemargin}{0.0in}
\setlength{\textwidth}{6.5in}
%\newenvironment{demo}[1][Demostraci\'on]{\noindent\textbf{#1.} }{\ \rule{0.5em}{0.5em}}

%\usepackage[left=1in,right=1in,top=1in,bottom=1in]{geometry}

%\usepackage{enumerate}
%\renewcommand{\theenumi}{\roman{enumi}}
%\renewcommand{\labelenumi}{(\theenumi)}
\renewcommand\labelenumi{(\roman{enumi})}
\renewcommand\theenumi\labelenumi


%%%%TEOREMAS CON ESTILO plain
\newtheorem{teo}{Teorema}%[chapter]
\newtheorem{cor}[teo]{Corolario}
\newtheorem{prop}[teo]{Proposici\'on}
\newtheorem{problem}[teo]{Problema}
\newtheorem{ejer}[teo]{Ejercicio}
\newtheorem{lema}[teo]{Lema}
\newtheorem{conjetura}[teo]{Conjetura}
\newtheorem{pregunta}[teo]{Pregunta}
\newtheorem*{teo2}{Teorema}
\newtheorem*{pregunta2}{Pregunta}
\newtheorem*{lema2}{Lema}

%%%%TEOREMAS CON ESTILO definition
\theoremstyle{definition}
\newtheorem{defi}[teo]{Definici\'on}
\newtheorem{ejem}[teo]{Ejemplo}
%\newtheorem{ejems}[teo]{Ejemplos}
\newtheorem{obs}[teo]{Observaci\'on}
\newtheorem{obss}[teo]{Observaciones}
\newtheorem{ntn}[teo]{Notaci\'on}
\newtheorem{convencion}[teo]{Convenci\'on}
\newtheorem{nota}[teo]{Nota}
\newtheorem{inter}[teo]{Interpretaci\'on}


%%Teoremas personalizados
\newtheorem{innercustomgeneric}{\customgenericname}
\providecommand{\customgenericname}{}
\newcommand{\newcustomtheorem}[2]{%
	\newenvironment{#1}[1]
	{%
		\renewcommand\customgenericname{#2}%
		\renewcommand\theinnercustomgeneric{##1}%
		\innercustomgeneric
	}
	{\endinnercustomgeneric}
}

\newcustomtheorem{customteo}{Teorema}
\newcustomtheorem{customax}{Axioma}
\newcustomtheorem{customprop}{Proposici\'on}
\newcustomtheorem{customlema}{Lema}
\newcustomtheorem{customnota}{Nota}
\newcustomtheorem{customcor}{Corolario}
\newcustomtheorem{customdef}{Definici\'on}
%\newcustomtheorem{customlemma}{Lemma} Es otro ejemplo
%Se llama como
%	\begin{customthm}{8}\label{eight}
%		Every theorem must be numbered by hand.
%	\end{customthm}
%	
%	Here is a reference to theorem~\ref{eight} and
%	one to the important lemma~\ref{life-universe-everything}
%	
%	\begin{customlemma}{42}\label{life-universe-everything}
%		This lemma explains everything.
%	\end{customlemma}
%
%
%%%Ver https://tex.stackexchange.com/questions/53978/custom-theorem-numbering
% para m\'as detalles
%%Tambi\'en se podr\'ia seguir lo dicho en
% https://tex.stackexchange.com/questions/64272/change-label-of-a-single-theorem-using-ntheorem


%%% Comandos adicionales
%\renewcommand{\chaptername}{Cap\'itulo}
%\renewcommand{\contentsname}{Contenido}
%\renewcommand{\indexname}{{\bf \'Indice}}

\newcommand{\N}{\mathbb{N}} %%%NATURALES
\newcommand{\Z}{\mathbb{Z}}  %%%%%ENTEROS
\newcommand{\C}{\mathbb{C}}  %%%COMPLEJOS
\newcommand{\R}{\mathbb{R}}  %%%REALES
\newcommand{\Q}{\mathbb{Q}}  %%%RACIONALES
\newcommand{\Hi}{\mathcal{H}}  %%%Espacio de Hilbert


\usepackage{calrsfs} %Modifica estilos caligraficos y permite declarar otros
\DeclareMathAlphabet{\pazocal}{OMS}{zplm}{m}{n}
%\DeclareMathAlphabet{\mathcalligra}{T1}{calligra}{m}{n}

\newcommand{\mb}[1]{\mathbb{ #1 }}  %%%%% ESTILO COMO LOS REALES
\newcommand{\md}[1]{\mathds{ #1 }} %%%% Estilo como los reales, incluye n\'umeros. Usa paquete dsfont
\newcommand{\mk}[1]{\pazocal{ #1 }} %%%% ESTILO CALigrafICO usual por el alfabeto pazocal que aparece arriba
\newcommand{\mc}[1]{\mathcal{ #1 }} %%%%Otro estilo caligrafico porque uso el paquete calrsfs
%Mas info en https://tex.stackexchange.com/questions/69085/two-different-calligraphic-font-styles-in-math-mode
\newcommand{\mr}[1]{\mathrm{ #1 }} %%%%Otro estilo roman

\newcommand{\gh}[2]{\pi_{ #1 }\paren{ #2 }}  %%%%Grupo de homotop\'ia
\newcommand{\gen}[1]{\langle #1 \rangle} %%%%% Grupo/espacio  generado por...

\newcommand{\paren}[1]{\left( #1 \right)}   %%%%%PARENTESIS
\newcommand{\abs}[1]{\left| #1 \right|}  %%%VALOR ABSOLUTO
%\renewcommand{\to}{\longrightarrow} %%%FLECHA LARGA
\renewcommand{\epsilon}{\varepsilon}  %%%%% Epsilon
\renewcommand{\emptyset}{\varnothing}  %%% CONJUNTO vacio
\renewcommand{\tilde}{\widetilde}  %%% TILDE
\renewcommand{\phi}{\varphi}%%%%%%%%%% phi
%\renewcommand{\rho}{\varrho}%%%%%%%%%% rho



%%%           Operadores matem\'aticos
\DeclareMathOperator*{\est}{St}%%% Por el * se comportan como \lim, es decir, aceptan l\'imites superiores e inferires
\DeclareMathOperator{\sop}{supp}
\DeclareMathOperator{\GL}{GL}
\DeclareMathOperator{\SL}{SL}
\DeclareMathOperator{\PSL}{PSL}
\DeclareMathOperator{\PO}{PO}
\DeclareMathOperator{\PU}{PU}
\DeclareMathOperator{\SO}{SO}
\DeclareMathOperator{\SU}{SU}
\DeclareMathOperator{\oo}{O}
\DeclareMathOperator{\uu}{U}
\DeclareMathOperator{\CAT}{CAT}
\DeclareMathOperator{\Top}{Top}
%\DeclareMathOperator{\pos}{PoSet}
%\DeclareMathOperator{\cco}{Csim_{o}}
%\DeclareMathOperator{\sset}{SSet}
%\DeclareMathOperator{\TAS}{T_0Aspaces}
%\DeclareMathOperator{\cat}{Cat}
%\DeclareMathOperator{\tcat}{TopCat}
%\DeclareMathOperator{\sspa}{SSpaces}
\DeclareMathOperator*{\id}{Id}
\DeclareMathOperator{\expo}{exp}
\DeclareMathOperator{\inte}{int}
\DeclareMathOperator{\ext}{ext}
\DeclareMathOperator{\fr}{Fr}
\DeclareMathOperator{\sen}{sen}
\DeclareMathOperator{\senh}{senh}
\DeclareMathOperator{\rot}{Rot}
\DeclareMathOperator{\diam}{di\acute{a}m}
\DeclareMathOperator{\area}{\acute{a}rea}
\DeclareMathOperator{\dive}{div}


%% Fin del preámbulo de ASG


\usepackage{framed,color}
\definecolor{shadecolor}{RGB}{248,248,248}

\renewcommand{\textfraction}{0.05}
\renewcommand{\topfraction}{0.8}
\renewcommand{\bottomfraction}{0.8}
\renewcommand{\floatpagefraction}{0.75}

\renewenvironment{quote}{\begin{VF}}{\end{VF}}
\let\oldhref\href
\renewcommand{\href}[2]{#2\footnote{\url{#1}}}

\makeatletter
\newenvironment{kframe}{%
\medskip{}
\setlength{\fboxsep}{.8em}
 \def\at@end@of@kframe{}%
 \ifinner\ifhmode%
  \def\at@end@of@kframe{\end{minipage}}%
  \begin{minipage}{\columnwidth}%
 \fi\fi%
 \def\FrameCommand##1{\hskip\@totalleftmargin \hskip-\fboxsep
 \colorbox{shadecolor}{##1}\hskip-\fboxsep
     % There is no \\@totalrightmargin, so:
     \hskip-\linewidth \hskip-\@totalleftmargin \hskip\columnwidth}%
 \MakeFramed {\advance\hsize-\width
   \@totalleftmargin\z@ \linewidth\hsize
   \@setminipage}}%
 {\par\unskip\endMakeFramed%
 \at@end@of@kframe}
\makeatother

%\renewenvironment{Shaded}{\begin{kframe}}{\end{kframe}}

\usepackage{makeidx}
\makeindex

\urlstyle{tt}

\usepackage{amsthm}
\makeatletter
\def\thm@space@setup{%
  \thm@preskip=8pt plus 2pt minus 4pt
  \thm@postskip=\thm@preskip
}
\makeatother

\frontmatter
\usepackage[]{natbib}
\bibliographystyle{apalike}

\title{Cálculo}
\author{ASG}
\date{2021-02-15}

\usepackage{amsthm}
\newtheorem{theorem}{Theorem}[chapter]
\newtheorem{lemma}{Lemma}[chapter]
\newtheorem{corollary}{Corollary}[chapter]
\newtheorem{proposition}{Proposition}[chapter]
\newtheorem{conjecture}{Conjecture}[chapter]
\theoremstyle{definition}
\newtheorem{definition}{Definition}[chapter]
\theoremstyle{definition}
\newtheorem{example}{Example}[chapter]
\theoremstyle{definition}
\newtheorem{exercise}{Exercise}[chapter]
\theoremstyle{remark}
\newtheorem*{remark}{Remark}
\newtheorem*{solution}{Solution}
\begin{document}
\maketitle

% you may need to leave a few empty pages before the dedication page

%\cleardoublepage\newpage\thispagestyle{empty}\null
%\cleardoublepage\newpage\thispagestyle{empty}\null
%\cleardoublepage\newpage
\thispagestyle{empty}

\begin{center}
Al futuro,

cuando al fin usaremos esto con fluidez.
%\includegraphics{images/dedication.pdf}
\end{center}

\setlength{\abovedisplayskip}{-5pt}
\setlength{\abovedisplayshortskip}{-5pt}

{
\hypersetup{linkcolor=}
\setcounter{tocdepth}{2}
\tableofcontents
}
\listoftables
\listoffigures
\hypertarget{prefacio}{%
\chapter*{Prefacio}\label{prefacio}}


Algo.

\hypertarget{agradecimientos}{%
\section*{Agradecimientos}\label{agradecimientos}}


A las personas que hicieron sencillo el publicar libros.

\begin{flushright}
ASG
\end{flushright}

\hypertarget{quiuxe9n-escribe}{%
\chapter{Quién escribe?}\label{quiuxe9n-escribe}}

Por ahora, solo ASG. Más tarde espero que más personas.

\mainmatter

\hypertarget{introducciuxf3n}{%
\chapter{Introducción}\label{introducciuxf3n}}

Si esto funciona, el sitio se actualizará automáticamente. Y la escritura se simplificará. Además, cada capítulo tendrá el enlace a su propio video sin problemas.

\hypertarget{ejemplo-01}{%
\chapter{Ejemplo 01}\label{ejemplo-01}}

En esta sesión veremos una consecuencia importante del Teorema de Green: si \(F\) es un campo de clase \(C^1\), el rotacional \(\mathop{\mathrm{Rot}}\left( F(\overline{x}) \right)\) de \(F\) en un punto \(\overline{x}\) se puede calcular como un límite sobre regiones muy generales. Antes de ello, es conveniente recordar que si \(A\subset\mathbb{R}^2\) es acotado, entonces

\begin{equation*}
        \mathop{\mathrm{di\acute{a}m}}(A) = \sup \left\{ \|\overline{x} - \overline{y}\| \left\lvert \ \overline{x},\ \overline{y}\in A \right.  \  \right\} .
\end{equation*}

\begin{proposition}
\protect\hypertarget{prp:prop01}{}{\label{prp:prop01} }Sean \(U\subset\mathbb{R}^2\) una región, \(F = \left(P, Q\right): U\to \mathbb{R}^2\) de clase \(C^1\) en \(U\), \(\overline{x}\in U\) y \(\left\{ \Omega_{\epsilon} \right\}_{0<\epsilon <c}\) una familia de subconjuntos de \(U\) que son Jordan--medibles, cerrados y acotados tales que \(\Gamma_{\epsilon} = \partial \Omega_{\epsilon} = \mathop{\mathrm{Fr}}\left( \Omega_{\epsilon}\right)\) es una curva cerrada simple, \(\overline{x}\in\mathop{\mathrm{int}}\left(\Omega_{\epsilon}\right)\) para toda \(0 < \epsilon < c\) y \(\lim\limits_{\epsilon\to 0^{+}} \mathop{\mathrm{di\acute{a}m}}\left(\Omega_{\epsilon}\right) = 0\). Entonces

\begin{equation*}
            \mathop{\mathrm{Rot}}\left( F(\overline{x})\right) = \lim\limits_{\epsilon\to 0^{+}} \frac{\displaystyle\int\limits_{\Gamma_{\epsilon}} F\cdot d\gamma_{\epsilon}}{\mathop{\mathrm{\acute{a}rea}}\left( \Omega_{\epsilon}\right)}
\end{equation*}

donde \(\gamma_{\epsilon}\) es una parametrización de \(\Gamma_{\epsilon}\) que la recorre en el sentido contrario al de las manecillas del reloj.
\end{proposition}

\begin{proof}

Notamos que para cada \(0< \epsilon<c\) se cumplen las hipótesis del Teorema de Green, por lo cual

\begin{equation*}
            \displaystyle \int\limits_{\Gamma_{\epsilon}} F\cdot d\gamma_{\epsilon} = \int\limits_{\Omega_{\epsilon}} \left( \frac{\partial Q}{\partial x} - \frac{\partial P}{\partial y} \right).
\end{equation*}

Luego, para cada \(\epsilon\), por el Teorema del Valor Promedio para integrales sobre conjuntos Jordan--medibles (\textcolor{blue}{¿recuerda dicho resultado?}), existe \(\overline{\xi}_{\epsilon}\in \Omega_{\epsilon}\) tal que

\begin{align*}
            \int\limits_{\Omega_{\epsilon}} \left( \frac{\partial Q}{\partial x} - \frac{\partial P}{\partial y} \right) & = \left( \frac{\partial Q}{\partial x}\left(\overline{\xi}_{\epsilon}\right) - \frac{\partial P}{\partial y}\left(\overline{\xi}_{\epsilon}\right) \right) \cdot m\left( \Omega_{\epsilon}\right) \\
            & = \left( \frac{\partial Q}{\partial x}\left(\overline{\xi}_{\epsilon}\right) - \frac{\partial P}{\partial y}\left(\overline{\xi}_{\epsilon}\right) \right) \cdot \mathop{\mathrm{\acute{a}rea}}\left( \Omega_{\epsilon}\right).
\end{align*}

Ya que \(\overline{x}\in\mathop{\mathrm{int}}\left(\Omega_{\epsilon}\right)\), entonces \(\mathop{\mathrm{int}}\left(\Omega_{\epsilon}\right)\neq\emptyset\), lo cual implica que \(\mathop{\mathrm{\acute{a}rea}}\left(\Omega_{\epsilon}\right)\neq 0\), de donde, al combinar las dos igualdades anteriores, para toda \(\epsilon\) se cumple que

\begin{equation*}
        \frac{\displaystyle\int\limits_{\Gamma_{\epsilon}} F\cdot d\gamma_{\epsilon}}{\mathop{\mathrm{\acute{a}rea}}\left(\Omega_{\epsilon}\right)} = \frac{\partial Q}{\partial x}\left(\overline{\xi}_{\epsilon}\right) - \frac{\partial P}{\partial y}\left( \overline{\xi}_{\epsilon}\right).
\end{equation*}

Para concluir, notamos que las funciones \(\frac{\partial Q}{\partial x}\) y \(\frac{\partial P}{\partial y}\) son continuas porque \(F\) es de clase \(C^1\), además,

\begin{equation*}
        \lim\limits_{\epsilon\to 0^{+}} \overline{\xi}_{\epsilon} = \overline{x}
\end{equation*}

porque \(\overline{x}\in\mathop{\mathrm{int}}\left(\Omega_{\epsilon}\right)\) para toda \(0<\epsilon < c\) y \(\lim\limits_{\epsilon\to 0^{+}} \mathop{\mathrm{di\acute{a}m}}\left(\Omega_{\epsilon}\right)=0\), entonces por un teorema de cambio de variable en límites (\textcolor{blue}{¿conoce algún teorema así?}) obtenemos que

\begin{align*}
        \displaystyle\lim\limits_{\epsilon\to 0^{+}} \frac{\displaystyle\int\limits_{\Gamma_{\epsilon}} F\cdot d\gamma_{\epsilon}}{\mathop{\mathrm{\acute{a}rea}}\left(\Omega_{\epsilon}\right)}  & = \lim\limits_{\epsilon\to 0^{+}}\left(\frac{\partial Q}{\partial x}\left(\overline{\xi}_{\epsilon}\right) - \frac{\partial P}{\partial y}\left( \overline{\xi}_{\epsilon}\right)\right) \\
        & = \frac{\partial Q}{\partial x}\left(\overline{x}\right) - \frac{\partial P}{\partial y}\left(\overline{x}\right) \\
        & = \mathop{\mathrm{Rot}}\left( F(\overline{x})\right).
\end{align*}

Esto termina la prueba.

\end{proof}

Aunque no lo parezca, la Proposición \ref{prp:prop01} se puede interpretar como la \textbf{rotación promedio} generada por el campo \(F\) en el punto \(\overline{x}\).

Para concluir esta sesión presentamos una aplicación directa del Teorema de Green: el cálculo de una integral.

\begin{exercise}
\protect\hypertarget{exr:ej01}{}{\label{exr:ej01} } Calcule la integral \(\displaystyle\int\limits_{\Gamma} F\cdot d\gamma\), donde \(F(x,y) = \left(3x^3-y^3, x^3 +2y^3\right)\) y \(\Gamma\) es el círculo unitario con centro en el origen y recorrido en el sentido contrario al de las manecillas del reloj.
\end{exercise}

\begin{proof}
\iffalse{} {Proof. } \fi{}Tenemos que \(F\) es de clase \(C^1\) en \(\mathbb{R}^2\) y si \(\Omega = B_1(0,0)\), entonces \(\Omega\) es un conjunto Jordan--medible y \(\mathop{\mathrm{Fr}}\left(\Omega\right) = \partial \Omega = \Gamma\). Como \(\mathbb{R}^2\) es una región, claramente se tiene que \(\Omega \cup \Gamma\subset \mathbb{R}^2\). Por lo tanto se cumplen las hipótesis del Teorema de Green, así que
\begin{equation*}
            \displaystyle\int\limits_{\Gamma=\partial \Omega} F \cdot d\gamma = \int\limits_{\Omega} \mathop{\mathrm{Rot}}\left( F \right), 
\end{equation*}
donde \(\gamma\) es la parametrización usual de \(\Gamma\) que la recorre una vez en el sentido contrario al de las manecillas del reloj.

Notamos que
\begin{equation*}
            \frac{\partial Q}{\partial x}(x,y) = 3x^2
\end{equation*}
y también
\begin{equation*}
            \frac{\partial P}{\partial y}(x,y) = -3y^2,
\end{equation*}
por lo cual
\begin{equation*}
            \mathop{\mathrm{Rot}}(F)(x,y) = 3x^2 + 3y^2 = 3(x^2 + y^2). 
\end{equation*}

En virtud de lo anterior, tenemos que
\begin{equation*}
            \displaystyle\int\limits_{\Gamma=\partial \Omega} F \cdot d\gamma = 3\int\limits_{B_1(0,0)} f
        \end{equation*}
donde \(f(x,y) = x^2+y^2\).

Notamos que al aplicar el cambio de variable a coordenadas polares obtenemos que
\begin{align*}
            \int\limits_{B_1(0,0)} f & = \int\limits_{0}^{2\pi}\left(\int\limits_{0}^{1} r^3 dr\right)\ d\theta \\
            & = \int\limits_{0}^{2\pi} \frac{1}{4} d\theta \\
            & = \frac{\pi}{2}.
        \end{align*}

Por lo tanto,
\begin{equation*}
            \displaystyle\int\limits_{\Gamma=\partial \Omega} F \cdot d\gamma = \frac{3\pi}{2}.
        \end{equation*}
\end{proof}

\bigskip

\textcolor{red}{¿Obtiene el mismo resultado si hace los cálculos directamente?}

\cleardoublepage

\hypertarget{appendix-apuxe9ndice-de-prueba}{%
\appendix \addcontentsline{toc}{chapter}{\appendixname}}


\hypertarget{por-aprender}{%
\chapter{Por aprender}\label{por-aprender}}

Este sería el final. Ver la Proposición \ref{prp:prop01}.

  \bibliography{book.bib,packages.bib}

\backmatter
\printindex

\end{document}
