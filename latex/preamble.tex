\usepackage{booktabs}
\usepackage{longtable}
\usepackage[bf,singlelinecheck=off]{caption}
\usepackage[scale=.8]{sourcecodepro}

%%%%%%%%%%%%%%%%%%%%%%%%%%%%%%%%%%%
%% Preámbulo usual de ASG (hace falta corregir duplicidades)

\usepackage{verbatim}
%%Los dos paquetes siguientes sirven para usar documentos de una sola ecuació\'on y para importarlos
\usepackage[subpreambles=true]{standalone}
\usepackage{import}
%%%Para m\'as detalles ver
%% https://www.overleaf.com/learn/latex/Multi-file_LaTeX_projects#The_standalone_package

%\usepackage{movie15}%para insertar videos
%ver https://tex.stackexchange.com/questions/7602/how-to-add-a-gif-file-to-my-latex-file

\usepackage{xcolor}
%\usepackage[utf8]{inputenc}
\usepackage[spanish]{babel}
\decimalpoint %para poner el punto decimal (en lugar de la coma decimal)
\usepackage{amsmath}
\allowdisplaybreaks% para permitir rompimiento de ecuaciones en p\'aginas distintas
% ver http://tex.stackexchange.com/questions/51682/is-it-possible-to-pagebreak-aligned-equations
% para m\'as detalles
\usepackage{amsthm}
%\swapnumbers%%%%%Para invertir el orden # Teorema en lugar de Toerema #, se corresponde con el paquete amsthm
\usepackage{dsfont}
\usepackage{amsfonts}
\usepackage{amssymb}%
\setcounter{MaxMatrixCols}{30}%
\usepackage{multicol}
\usepackage{graphicx}
\usepackage{cases}
\providecommand{\U}[1]{\protect\rule{.1in}{.1in}}

\setlength{\topmargin}{-1.3in}
\setlength{\textheight}{8.75in}
\setlength{\headheight}{0.75in}
\setlength{\oddsidemargin}{0.0in}
\setlength{\evensidemargin}{0.0in}
\setlength{\textwidth}{6.5in}
%\newenvironment{demo}[1][Demostraci\'on]{\noindent\textbf{#1.} }{\ \rule{0.5em}{0.5em}}

\usepackage[left=1in,right=1in,top=1in,bottom=1in]{geometry}

%\usepackage{enumerate}
%\renewcommand{\theenumi}{\roman{enumi}}
%\renewcommand{\labelenumi}{(\theenumi)}
\renewcommand\labelenumi{(\roman{enumi})}
\renewcommand\theenumi\labelenumi


%%%%TEOREMAS CON ESTILO plain
\newtheorem{teo}{Teorema}%[chapter]
\newtheorem{cor}[teo]{Corolario}
\newtheorem{prop}[teo]{Proposici\'on}
\newtheorem{problem}[teo]{Problema}
\newtheorem{ejer}[teo]{Ejercicio}
\newtheorem{lema}[teo]{Lema}
\newtheorem{conjetura}[teo]{Conjetura}
\newtheorem{pregunta}[teo]{Pregunta}
\newtheorem*{teo2}{Teorema}
\newtheorem*{pregunta2}{Pregunta}
\newtheorem*{lema2}{Lema}

%%%%TEOREMAS CON ESTILO definition
\theoremstyle{definition}
\newtheorem{defi}[teo]{Definici\'on}
\newtheorem{ejem}[teo]{Ejemplo}
%\newtheorem{ejems}[teo]{Ejemplos}
\newtheorem{obs}[teo]{Observaci\'on}
\newtheorem{obss}[teo]{Observaciones}
\newtheorem{ntn}[teo]{Notaci\'on}
\newtheorem{convencion}[teo]{Convenci\'on}
\newtheorem{nota}[teo]{Nota}
\newtheorem{inter}[teo]{Interpretaci\'on}


%%Teoremas personalizados
\newtheorem{innercustomgeneric}{\customgenericname}
\providecommand{\customgenericname}{}
\newcommand{\newcustomtheorem}[2]{%
	\newenvironment{#1}[1]
	{%
		\renewcommand\customgenericname{#2}%
		\renewcommand\theinnercustomgeneric{##1}%
		\innercustomgeneric
	}
	{\endinnercustomgeneric}
}

\newcustomtheorem{customteo}{Teorema}
\newcustomtheorem{customax}{Axioma}
\newcustomtheorem{customprop}{Proposici\'on}
\newcustomtheorem{customlema}{Lema}
\newcustomtheorem{customnota}{Nota}
\newcustomtheorem{customcor}{Corolario}
\newcustomtheorem{customdef}{Definici\'on}
%\newcustomtheorem{customlemma}{Lemma} Es otro ejemplo
%Se llama como
%	\begin{customthm}{8}\label{eight}
%		Every theorem must be numbered by hand.
%	\end{customthm}
%	
%	Here is a reference to theorem~\ref{eight} and
%	one to the important lemma~\ref{life-universe-everything}
%	
%	\begin{customlemma}{42}\label{life-universe-everything}
%		This lemma explains everything.
%	\end{customlemma}
%
%
%%%Ver https://tex.stackexchange.com/questions/53978/custom-theorem-numbering
% para m\'as detalles
%%Tambi\'en se podr\'ia seguir lo dicho en
% https://tex.stackexchange.com/questions/64272/change-label-of-a-single-theorem-using-ntheorem


%%% Comandos adicionales
%\renewcommand{\chaptername}{Cap\'itulo}
%\renewcommand{\contentsname}{Contenido}
%\renewcommand{\indexname}{{\bf \'Indice}}

\newcommand{\N}{\mathbb{N}} %%%NATURALES
\newcommand{\Z}{\mathbb{Z}}  %%%%%ENTEROS
\newcommand{\C}{\mathbb{C}}  %%%COMPLEJOS
\newcommand{\R}{\mathbb{R}}  %%%REALES
\newcommand{\Q}{\mathbb{Q}}  %%%RACIONALES
\newcommand{\Hi}{\mathcal{H}}  %%%Espacio de Hilbert


\usepackage{calrsfs} %Modifica estilos caligraficos y permite declarar otros
\DeclareMathAlphabet{\pazocal}{OMS}{zplm}{m}{n}
%\DeclareMathAlphabet{\mathcalligra}{T1}{calligra}{m}{n}

\newcommand{\mb}[1]{\mathbb{ #1 }}  %%%%% ESTILO COMO LOS REALES
\newcommand{\md}[1]{\mathds{ #1 }} %%%% Estilo como los reales, incluye n\'umeros. Usa paquete dsfont
\newcommand{\mk}[1]{\pazocal{ #1 }} %%%% ESTILO CALigrafICO usual por el alfabeto pazocal que aparece arriba
\newcommand{\mc}[1]{\mathcal{ #1 }} %%%%Otro estilo caligrafico porque uso el paquete calrsfs
%Mas info en https://tex.stackexchange.com/questions/69085/two-different-calligraphic-font-styles-in-math-mode
\newcommand{\mr}[1]{\mathrm{ #1 }} %%%%Otro estilo roman

\newcommand{\gh}[2]{\pi_{ #1 }\paren{ #2 }}  %%%%Grupo de homotop\'ia
\newcommand{\gen}[1]{\langle #1 \rangle} %%%%% Grupo/espacio  generado por...

\newcommand{\paren}[1]{\left( #1 \right)}   %%%%%PARENTESIS
\newcommand{\abs}[1]{\left| #1 \right|}  %%%VALOR ABSOLUTO
%\renewcommand{\to}{\longrightarrow} %%%FLECHA LARGA
\renewcommand{\epsilon}{\varepsilon}  %%%%% Epsilon
\renewcommand{\emptyset}{\varnothing}  %%% CONJUNTO vacio
\renewcommand{\tilde}{\widetilde}  %%% TILDE
\renewcommand{\phi}{\varphi}%%%%%%%%%% phi
%\renewcommand{\rho}{\varrho}%%%%%%%%%% rho



%%%           Operadores matem\'aticos
\DeclareMathOperator*{\est}{St}%%% Por el * se comportan como \lim, es decir, aceptan l\'imites superiores e inferires
\DeclareMathOperator{\sop}{supp}
\DeclareMathOperator{\GL}{GL}
\DeclareMathOperator{\SL}{SL}
\DeclareMathOperator{\PSL}{PSL}
\DeclareMathOperator{\PO}{PO}
\DeclareMathOperator{\PU}{PU}
\DeclareMathOperator{\SO}{SO}
\DeclareMathOperator{\SU}{SU}
\DeclareMathOperator{\oo}{O}
\DeclareMathOperator{\uu}{U}
\DeclareMathOperator{\CAT}{CAT}
\DeclareMathOperator{\Top}{Top}
%\DeclareMathOperator{\pos}{PoSet}
%\DeclareMathOperator{\cco}{Csim_{o}}
%\DeclareMathOperator{\sset}{SSet}
%\DeclareMathOperator{\TAS}{T_0Aspaces}
%\DeclareMathOperator{\cat}{Cat}
%\DeclareMathOperator{\tcat}{TopCat}
%\DeclareMathOperator{\sspa}{SSpaces}
\DeclareMathOperator*{\id}{Id}
\DeclareMathOperator{\expo}{exp}
\DeclareMathOperator{\inte}{int}
\DeclareMathOperator{\ext}{ext}
\DeclareMathOperator{\fr}{Fr}
\DeclareMathOperator{\sen}{sen}
\DeclareMathOperator{\senh}{senh}
\DeclareMathOperator{\rot}{Rot}
\DeclareMathOperator{\diam}{di\acute{a}m}
\DeclareMathOperator{\area}{\acute{a}rea}
\DeclareMathOperator{\dive}{div}


%% Fin del preámbulo de ASG


\usepackage{framed,color}
\definecolor{shadecolor}{RGB}{248,248,248}

\renewcommand{\textfraction}{0.05}
\renewcommand{\topfraction}{0.8}
\renewcommand{\bottomfraction}{0.8}
\renewcommand{\floatpagefraction}{0.75}

\renewenvironment{quote}{\begin{VF}}{\end{VF}}
\let\oldhref\href
\renewcommand{\href}[2]{#2\footnote{\url{#1}}}

\makeatletter
\newenvironment{kframe}{%
\medskip{}
\setlength{\fboxsep}{.8em}
 \def\at@end@of@kframe{}%
 \ifinner\ifhmode%
  \def\at@end@of@kframe{\end{minipage}}%
  \begin{minipage}{\columnwidth}%
 \fi\fi%
 \def\FrameCommand##1{\hskip\@totalleftmargin \hskip-\fboxsep
 \colorbox{shadecolor}{##1}\hskip-\fboxsep
     % There is no \\@totalrightmargin, so:
     \hskip-\linewidth \hskip-\@totalleftmargin \hskip\columnwidth}%
 \MakeFramed {\advance\hsize-\width
   \@totalleftmargin\z@ \linewidth\hsize
   \@setminipage}}%
 {\par\unskip\endMakeFramed%
 \at@end@of@kframe}
\makeatother

%\renewenvironment{Shaded}{\begin{kframe}}{\end{kframe}}

\usepackage{makeidx}
\makeindex

\urlstyle{tt}

\usepackage{amsthm}
\makeatletter
\def\thm@space@setup{%
  \thm@preskip=8pt plus 2pt minus 4pt
  \thm@postskip=\thm@preskip
}
\makeatother

\frontmatter
